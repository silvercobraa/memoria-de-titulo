\chapter{Conclusiones}

Se mostró que el Robot Baxter es capaz de resolver el cubo de Rubik en su totalidad, desde recogerlo, pasando por su detección y resolución hasta su manipulación, al menos para instancias pequeñas, y con algo de suerte en la precisión del robot. Esto se consiguió aplicando diversas técnicas de varias áreas de la inteligencia artificial.

El sistema desarrollado consigue detectar correctamente sobre el 90\% de los facelets individuales del cubo, pero aún hay mucho que se puede mejorar. Como trabajo futuro se puede explorar la posibilidad de usar visión para recoger el cubo de la mesa, independientemente de dónde esté ubicado; utilizar las cámaras de los brazos del robot para corregir de alguna manera los agarres mal alineados antes que sucedan; explorar otros esquemas de posicionamiento, de manera que se minimicen reduzcan los desplazamientos de los brazos y los cambios de mano; aplicar técnicas más sofisticadas de visión computacional para detectar el estado del cubo, sin ayuda de artefactos como los círculos o la goma EVA; considerar un solucionador óptimo o, que al menos genere soluciones más cortas que Kociemba en un tiempo razonable.
