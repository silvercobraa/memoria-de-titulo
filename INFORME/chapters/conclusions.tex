\chapter{Conclusiones}

Se demostró que el Robot Baxter es capaz de resolver el cubo de Rubik en su totalidad, desde recogerlo, pasando por su detección y resolución hasta su manipulación, al menos para instancias pequeñas, con algunas preparaciones previas y con algo de suerte en la precisión del robot. Esto se consiguió aplicando diversas técnicas de varias áreas de la inteligencia artificial.

El sistema desarrollado consigue detectar correctamente sobre el 90\% de los facelets individuales del cubo, pero aún hay mucho que se puede mejorar. Hay varios aspectos que se pueden trabajar a futuro:
\begin{itemize}
	\item Explorar la posibilidad de usar visión para recoger el cubo de la mesa, independientemente de dónde esté ubicado.
	\item Utilizar las cámaras de los brazos del robot para corregir de alguna manera los agarres mal alineados antes que sucedan.
	\item Explorar otros esquemas de posicionamiento, de manera que se minimicen o reduzcan los desplazamientos de los brazos y/o los cambios de mano.
	\item Aplicar técnicas más sofisticadas de visión computacional para detectar el estado del cubo, posiblemente sin ayuda de artefactos como los círculos o la goma EVA.
	\item Considerar un solucionador óptimo, o que, al menos genere soluciones más cortas que Kociemba en un tiempo razonable.
\end{itemize}


El proyecto realizado involucró utilizar un Robot para una tarea poco convencional. Se espera que este trabajo pueda inspirar y haga crecer el interés en la robótica en los estudiantes de Informática y carreras afines, que los haga explorar las capacidades de la robótica más allá de sus aplicaciones industriales típicas y que los incentive a resolver problemas complejos mediante la integración del mundo del software y del hardware.
