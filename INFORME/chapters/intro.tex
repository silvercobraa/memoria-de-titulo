\chapter{Introducción}

El cubo de Rubik es un puzzle mecánico, inventado por Erno Rubik en $1974$~\cite{ernorubik1974}. Consiste de un sistema de piezas móviles formadas al subdividir cada una de las caras de un cubo en una grilla de $3\times 3$ cuadrados de igual tamaño, permitiendo que aristas y esquinas se permuten entre sí al rotar alguna de las caras en torno a un eje que atraviesa el centro de dicha cara. Cada cara tiene un color diferente a las demás, y el objetivo del rompecabezas es, una vez permutado el cubo, llevarlo a una configuración donde todos los cuadrados (también llamados ``\emph{facelets}'') de una misma cara tengan el mismo color. Ésto debe conseguirse sólo mediante rotaciones de caras sobre sus ejes y sin algún tipo de trampa como destornillar los ejes o sacar los stickers de los facelets.

A pesar de su aparente simplicidad, este juguete oculta una complejidad que ha fascinado a científicos de la computación por décadas. Con nada más ni nada menos que $43\,252\,003\,274\,489\,856\,000$ configuraciones posibles\cite{mathematicsrubik}\cite{mathematicsrubik2}, la tarea de resolverlo no es para nada trivial. Varios algoritmos de diversa complejidad han sido desarrollados para solucionar el cubo desde entonces, cada uno con ventajas y desventajas.

\begin{figure}[ht]
	\centering
	\includegraphics[width=0.5\textwidth]{figures/baxter}
	\caption{El robot Baxter.}
	\label{baxter}
\end{figure}

El interés en resolver el cubo de Rubik no se detiene en el ámbito del software. Numerosos robots se han diseñado específicamente para resolver cubos, pero también se han utilizado robots con propósitos más generales para esta tarea. Uno de estos robots, es el robot Baxter.

Baxter es un robot humanoide, desarrollado por la compañía Rethink Robotics\cite{baxterproduct}. Mide entre $122$ y $190$ centímetros de alto, y pesa $138$ kilogramos. Posee $2$ brazos con $7$ articulaciones cada uno, $4$ de tipo rotatoria y $3$ ``codos'' intercaladas entre sí. Un par de pinzas paralelas o ``\emph{grippers}'' se puede adjuntar en los extremos de cada brazo, las cuales pueden abrirse o cerrarse con el propósito de manipular una variedad objetos. Estas pinzas vienen en diversas formas y tamaños. El robot también posee sensores para determinar la posición y torque de cada articulación en cada una de éstas. Además, cuenta con cámaras en sus $2$ brazos y en su cabeza. La cabeza de Baxter contiene una pantalla, en la cual se puede mostrar información relevante\cite{baxterspecs}.
Este robot ha sido utilizado ampliamente tanto en la industria como en la academia, realizando tareas desde manipular objetos en líneas de ensamblaje\cite{baxtersupply}, pasando por jugar ajedrez\cite{baxterchess} e incluso asistir a humanos en ejercicios diseñados para terapia física\cite{baxterassistive}.

El objetivo general de este trabajo es, conseguir que un robot Baxter arme un cubo de Rubik desde cero. Esto incluye los objetivos particulares de ($1$) ser capaz de recoger el cubo desde algún lugar, ($2$) utilizar sensores (en particular, cámaras) para detectar el estado del cubo, ($3$) solucionarlo, ($4$) planificar una secuencia de movimientos y ($5$) efectuarlos mediante el hardware del robot. Para esto se emplea el Robot Baxter ubicado en el Laboratorio FabLab de la Facultad de Ingeniería de la Universidad de Concepción, y un cubo de Rubik capaz de ``cortar esquinas'' (esto es, que sus caras puedan rotar aún cuando no estén perfectamente alineadas) y ligeramente preparado para su uso por el robot.

Como objetivos específicos se tiene aplicar técnicas de robótica para programar un robot para una tarea que requiere motricidad fina; aplicar técnicas de visión computacional para detectar el estado de un cubo de Rubik; utilizar métodos de inteligencia artificial en la resolución de un problema combinatorio y diseñar e implementar un sistema que integre todas estas tareas.

Al finalizar este proyecto, se logró conseguir que el robot Baxter detectara correctamente más del 90\% del total de los colores del cubo sobre todos los experimentos realizados, y más de la mitad de cada experimento individual en su totalidad. El robot es capaz de llevar el cubo de su estado inicial hasta su estado resuelto realizando la secuencia de giros correcta completamente, siempre y cuando dicha secuencia no sea demasiado larga.
