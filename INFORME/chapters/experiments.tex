\chapter{Experimentos y resultados}

Se realizaron 2 tipos de experimentos, de visión y de manipulación. Pruebas de resolución no se llevaron a cabo, ya que el solucionador siempre encuentra una solución, a menos que el cubo se encuentre en un estado inválido.

Se utilizó el scrambler legacy de la World Cube Association, para generar 20 permutaciones del cubo de Rubik. Cada una de estas permutaciones, también llamados ``desarmes'' consiste de 30 rotaciones, y fueron aplicadas manualmente sobre el cubo.

Para la visión, se utilizaron los




Dado que el espacio de estados del cubo de Rubik es enorme ($\approx 4.5*10^{19}$) ninguna cantidad no astronómica de pruebas será representativa.
