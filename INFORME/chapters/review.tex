\chapter{Discusión Bibliográfica}

\section*{Rubik}
Un cubo de Rubik válidamente permutado tiene una solución dada por una secuencia de rotaciones de caras que, al aplicarlas en orden, llevan el cubo a su estado resuelto. Usualmente, se utilizan 2 métricas para contar la cantidad de rotaciones: \emph{quarter-turn metric} o ``QTM'' y \emph{half-turn metric} o ``HTM''\cite{qtmhtm}. La primera considera cada rotación de $90$ grados (ya sea en sentido horario o antihorario) como una acción por sí misma. La segunda es similar, salvo que además considera los giros de $180$ grados como $1$ acción en lugar de $2$.

A lo largo de los años, se ha ido reduciendo la cota superior del largo máximo de una solución para cualquier configuración del cubo. A este valor se le conoce como el ``número de Dios'', en alusión a el ``algoritmo de Dios'', un algoritmo que sería capaz de resolver cualquier cubo de Rubik en la menor cantidad de movimientos.

En $1981$, Thistlethwaite\cite{thistlethwaite} demuestra que $52$ movimientos bastan para resolver cualquier cubo. Posteriormente, Kloosterman\cite{kloosterman} reduce esta cota a $42$ en $1990$. En $1992$, Reid y Winter disminuyen este valor a $39$ y $37$ movimientos, transcurriendo sólamente $1$ día entre estos dos eventos\cite{reid39}\cite{winter}. En $1995$, Reid baja aún mas la cota a $29$ giros \cite{reid29}.

En $1995$, Reid demuestra que la posición ``superflip'', una configuración en la que el cubo está resuelto, con la excepción de que todas las aristas están volteadas en su lugar, requiere de 20 rotaciones\cite{superflip} en HTM, dando una cota inferior para el largo de una solución en el peor caso.

La cota superior continuó descendiendo, demostrándose de $28$ en y $27$ por Radu\cite{radu28}\cite{radu27} en $2005$ y $2006$ respectivamente, $25$ por Kunkle\cite{kunkle} en $2007$ y $25$, $23$ y luego $22$ por Rokicki\cite{rokicki25}\cite{rokicki23}\cite{rokicki22} et al en $2008$.

Finalmente, en julio de $2010$ Rokicki, Kociemba, Davidson y Dethridge demuestran que el número de Dios es $20$\cite{godnumber20}.

En cuanto a métodos concretos para resolver el cubo, varias alternativas existen actualmente. El ``método del principiante''\cite{beginner} es el que suele ser utilizado por los cuberos novatos. Consiste en armar el cubo por niveles: primero se resuelve la capa inferior, luego la del medio y finalmente la superior. Esto se lleva a cabo utilizando secuencias de giros que preserven lo que ya ha sido armado en los pasos anteriores. Es un método amigable para humanos, ya que requiere memorizar pocas secuencias para ubicar correctamente todas las piezas del cubo, pero suele incurrir en pasos redundantes lo que implica soluciones largas.

En competencias de speedcubing, torneos donde gente intenta armar cubos de Rubik lo más rápido posible, otros métodos suelen ser utilizados, que favorecen el tiempo de resolución y reducen el largo de las soluciones, pero que requieren memorizar más secuencias de rotaciones. El método ``CFOP'' (abreviación inglesa de ``Cross, First 2 layers, Orientation last layer, Permutation last layer''), propuesto por Jessica Fridrich\cite{fridrich} y en menor medida el método ``Petrus'', propuesto por Lars Petrus\cite{petrus} son métodos que caen en esta categoría, pero que suelen ser considerados de dificultad intermedia o avanzada para humanos.

Otros métodos menos amigables para seres humanos se basan en búsqueda. Algunos funcionan descomponiendo el cubo de Rubik usando teoría de grupos en varios subgrupos para reducir el espacio de búsqueda, como es el caso del algoritmo de Thistlethwaite\cite{thistlethwaite} (5 subgrupos) y su versión mejorada, el algoritmo de Kociemba\cite{kociemba} (3 subgrupos). Otro enfoque es el algoritmo de Korf\cite{korf}, que se basa en una búsqueda informada con profundidad iterativa (IDA*)\cite{ida} y bases de datos de patrones\cite{patterndatabases}. Este último es una instancia del algoritmo de Dios, ya que entrega siempre la solución óptima, a cambio de un costo computacional tremendo, exponencial en el largo de la solución\cite{korfcomplexity}. DeepCubeA\cite{deepcube}, un método basado en aprendizaje por refuerzo profundo, es capaz de encontrar soluciones cortas sin conocimiento específico para el cubo de Rubik, y además generaliza a otros puzzles como el puzzle 15 y luces fuera.


\section*{Visión Computacional}
Si bien encontrar una solución para un cubo de Rubik se puede considerar un problema resuelto, hay muchos pasos previos que se deben realizar antes de llegar a una representación del estado apta para su resolución. Para obtener el estado del cubo, bien se puede ingresar manualmente (por ejemplo, mediante teclado) o se puede detectar automáticamente. Si se desea esto último, como es el caso de este proyecto, se deben aplicar técnicas de visión computacional.

La visión computacional es el área de la inteligencia artificial que estudia cómo automatizar las tareas realizadas por el sistema visual humano. En este campo se desarrollan métodos para que computadores puedan percibir, analizar, procesar e interpretar imágenes del mundo real\cite{vision}.

Algunos de los problemas que vuelven a la visión computacional una tarea muy compleja son la perspectiva, la orientación, las oclusiones y la deformación, pues pueden causar formas totalmente distintas para el mismo objeto. Otros aspectos dificultan la percepción de color, como las sombras, la reflexión especular y las interreflexiones\cite{aima}.

El problema de detectar los colores del cubo de Rubik puede atacarse con modelos de aprendizaje supervisado, pero aún con numerosos ejemplos es difícil generalizar a diferentes entornos, cambios en iluminación o desgaste del material \cite{rubikcolors}.

\section*{Robótica}
Un robot es un agente que realiza tareas en el mundo físico por medio de sus efectores\cite{aima}. La robótica es el estudio y diseño de estos agentes. La mayoría de los robots puede clasificarse como robot manipulador o robot móvil. Baxter es un robot del tipo manipulador, pues sus únicos efectores son sus brazos y su cabeza, siendo incapaz de desplazarse.

Un problema de importancia en robótica es el de las cinemáticas inversas\cite{ik} (o ``IK'', del inglés ``Inverse Kinematics''), que consiste en obtener coordenadas en el espacio de configuración propio del robot -que en el caso de Baxter, corresponde a encontrar los ángulos de cada articulación de un brazo- a partir de coordenadas cartesianas del espacio tridimensional. Esto no siempre es posible, ya que el mismo brazo del robot restringe las posiciones válidas. Por ejemplo, no es posible alcanzar puntos más allá del largo del brazo de Baxter, ni curvar los brazos más allá de los ángulos límite de las articulaciones, ni llegar a puntos que causen colisiones con sigo mismo. Algunas posiciones son alcanzables pero sólo en una determinada orientación, por ejemplo, un punto ubicado muy a la derecha del robot puede alcanzarse con el brazo izquierdo sólo orientado hacia la derecha. Estas restricciones juegan un papel importante al decidir las posturas que se le den al robot para manipular el cubo.

Existe un framework llamado ROS (Robot Operating System)\cite{ros} que contiene un conjunto de bibliotecas que simplifican la escritura de software para robots.
Para programar al robot Baxter, se requiere como dependencia a ROS. De ésta manera, Baxter cuenta un servicio de cinemáticas inversas, pero también es posible especificar los ángulos de cada articulación directamente.


\section*{Baxter armando el cubo de Rubik}
Algunos trabajos previos que intentan armar el cubo de Rubik con el Robot Baxter son los de ActiveRobots y proyectos de estudiantes universitarios. El primero \cite{baxterrubik1}, utiliza un elemento de hardware adicional para escanear el cubo, el cual no se encuentra disponible en la Universidad de Concepción. El trabajo de \cite{baxterrubik3} sólo es capaz de rotar las caras izquierda y derecha del cubo. El trabajo de \cite{baxterrubik2} no emplea visión computacional para la detección de estados.

Existen varios trabajos de robots armando el cubo de Rubik, en particular, destaca el trabajo de \cite{mitrubikrobot}, capaz de armar un cubo de Rubik en 0.38 segundos, superando el récord mundial humano. Sin embargo, este requiere un hardware altamente especializado, mientras que los demás trabajos como el de \cite{robotother} o \cite{mrroc} utilizan otros Robots diferentes de Baxter, los cuales no se encuentran disponibles en la Universidad de Concepción.
